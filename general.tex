\documentclass[preprint]{aastex63}


\usepackage[english]{babel}
\usepackage[utf8]{inputenc}
\usepackage{amsmath,amssymb}
\usepackage{parskip}
\usepackage{graphicx}
\usepackage{fancyhdr}



% Margins

%%%%%%%%%%%%%%%%%
%     Title     %
%%%%%%%%%%%%%%%%%

\begin{document}

\title{THE SWIFT AGN \& CLUSTER SURVEY V: MEASURING THE AGN ANGULAR CORRELATION FUNCTION}
\author{Brett Bonine}
\affiliation{Homer L. Dodge Department of Physics \& Astronomy, University of Oklahoma, Norman, OK 73019}

\begin{abstract} 
    We present a measurement of the angular clustering of 22,000 X-ray selected AGN from the SWIFT AGN and Cluster survey (SACS),
    one of the largest medium-depth serendipitous X-Ray surveys to date. 
    To accomplish this, we measured the angular correlation function against random source distributions simulated based on SWIFT XRT exposure maps for 739 fields and then calculated the weighted-average correlation function
    over all fields. We investigate differences between the clustering of obscured and unobscured AGN by 
    using the  MIR blue and MIR red subsamples based on their available \textit{WISE} colors. Finally, we deproject the angular correlation function 
    via Limber's Inversion to infer the real-space clustering properties and the AGN bias.
\end{abstract}

\keywords{cosmology: large-scale structure of the universe; galaxies: active; surveys; X-rays: galaxies}
%%%%%%%%%%%%%%%%%%%%%%%%%%%%%%%%%%%%%%%%%%%%%%%%%%%%%%
% TITLE PAGE 
%%%%%%%%%%%%%%%%%%%%%%%%%%%%%%%%%%%%%%%%%%%%%%%%%%%%%%%
\begin{titlepage}
\center
\Large{\textbf{THE SWIFT AGN \& CLUSTER SURVEY V:}} \\
MEASURING THE AGN ANGULAR CORRELATION FUNCTION \\
\bigskip
\bigskip
\bigskip
\bigskip
\Large{\textbf{Brett Bonine}}

\vfill
\begin{figure}[!h]
    
    \centering
    \includegraphics[scale = 0.15]{figs/ou_logo.png}
    
\end{figure}

\vfill
The University of Oklahoma

Homer L. Dodge Department of Physics \& Astronomy

\vfill

\textit{A GENERAL EXAMINATION submitted in fulfillment of the requirements for the DOCTORAL DEGREE}
\end{titlepage}

%%%%%%%%%%%%%%%%%%%%%%%%%%%%%%%%%%%%%%%%%%%%%%%%%%%%%%
% MAIN TEXT
%%%%%%%%%%%%%%%%%%%%%%%%%%%%%%%%%%%%%%%%%%%%%%%%%%%%%%%
\newpage
\setcounter{page}{1}

\section{INTRODUCTION}
The origin and evolution of large scale structure in the universe 
remains at the forefront of modern astronomy. As our own solar system is embedded in a galaxy, it is crucial
to understand how galaxies such as our own came to be. The finite travel time of light from distant regions of the universe 
gives astronomers a unique ability to look into the past: light we observe now from structures at cosmological distances
was emitted far in the ancient history of the universe. Due to the expansion of the universe, light emitted from these 
distant objects acquires a characteristic redshift defined to be

\begin{align}
    \label{eq:redshift}
    z = \frac{\lambda_{obs}-\lambda_{rest}}{\lambda_{rest}}
\end{align}
Where $\lambda_{obs}$ and $\lambda_{rest}$ are the observed and rest-frame wavelengths of light, respectively. The metric expansion of the universe
allows us to relate the observed redshift of an object to its distance, and therefore the age of the universe
when the light from that object was emitted. Thus, we can study the evolution of structure in the universe
by comparing the behavior of structures at progressively higher redshifts.


It has been well established that the majority of large galaxies contain a supermassive black hole (SMBH)
at their center \citep{magorrian}. Many studies have found correlations between SMBH properties and properties of their 
host galaxies, such as a correlation between SMBH mass and host galaxy bulge mass \citep{mcconnel} and the famous $M-\sigma$ relation that correlates SMBH mass with 
the velocity dispersion of stars in the host galaxy \citep{gebhardt}. For this reason, studying SMBHs can give insight into the growth 
and evolution of their host galaxies.  Whenever gas and dust within a galaxy make their way to the central SMBH, inter-particle interactions within the 
in-falling material causes emission across a broad range of wavelengths. These ``active'' galactic nuclei (AGN) can 
therefore be used as signposts for galaxy growth due to their incredible luminosity. 
 Two of the proposed mechanisms for AGN triggering are major mergers (where material can experience gravitational torque from the merger of two galaxies and lose angular momentum) and disk instability (gas inflow due to the instability 
inherent to a self-gravitating gas) \citep{oogi}. The former mechanism  has been thoroughly investigated in the literature and found to be consistent with observations by many analyses, but inconsistent by some others. 
\citet{hine} investigated the rest-frame UV merger fraction in a $z\approx 3$ protocluster found by \citet{lehmer} to have an overdensity of AGN and determined an elevated fraction of merging galaxies compared to the field. Similarly, \citet{fan} and \citet{glikman} 
also found elevated merger fractions by studying quasar morphology and brightness profiles. On the contrary,  some observations of AGN with moderate X-ray luminosity ($L_x \approx 10^{43 erg} \ s^{-1}$) have failed to find evidence of morphologies consistent with major mergers \citep{kocevski,cisternas:2011} 
or enhanced merger fractions of AGN compared to the field \citep{marian:2019}. Thus, the exact mechanism(s) that cause material to migrate to the central SMBH and trigger AGN remains an area of ongoing research. 


\section{Previous Work} 
Due to the importance of clustering measurements in constraining AGN formation channels, many studies have previously explored the clustering properties of AGN. 
A direct probe of the AGN environment is the two-point spatial correlation function, where redshift measurements for each AGN are
combined with their angular positions on the sky to measure their three-dimensional clustering. Large area optical band redshift surveys such as the Sloan Digital Sky Survey (SDSS) 
have allowed for precise measurements of the spatial clustering of optically bright luminous AGN  \citep{ross:2009}. 

Because a large volume of redshift measurements are required to measure the spatial correlation function of a sample, many studies instead opt 
to measure the \textit{angular} correlation function and deproject it using Limber's inversion \citep{limber} by making several assumptions about 
the distribution of true AGN population. 
The observed clustering properties can either be related to the typical environment of AGN by measuring the AGN bias \citep[see below]{tinker:2005} or compared with the predictions from semi-analytic models of various possible AGN environments.  
Angular correlation measurements of AGN have been facilitated in recent decades with the help of large-scale X-ray surveys from space-based observatories 
such as \textit{XMM-Newton}, \textit{Chandra} and \textit{ROSAT}. \citet{elyiv,ebrero,gandhi} inferred the characteristic angular clustering amplitude and slope 
in both the soft and hard band from serendipitous \textit{XMM-Newton} surveys. \citet{koutoulidis} investigated AGN clustering from hard band \textit{Chandra} data. \citet{miyaji} used an analogous 
technique involving cross-correlation with galaxies from SDSS. A summary of the measurements made by existing X-ray surveys in the literature 
is summarized below in Fig. \ref{fig:comp}. Although measurements of the clustering amplitude $\theta_0$ differ within the literature, 
clustering measurements of X-ray selected AGN consistently find a higher value of the power-law slope than the 
canonical value of $\gamma = 1.8$ that has been well established by optical surveys \citep{peebles:1980}.

The clustering AGN have been used to investigate the evolution of AGN properties over cosmic time \citep{koutoulidis_2},
the formation of SMBHs \citep{miyaji}, and the distribution of density fluctuations at various scales. Of particular interest to this work is the ability to test AGN unification theories
by comparing the clustering of obscured and unobscured AGN \citep{ebrero}. 
Despite assumptions made by the Limber integral equation, \citet{koutoulidis} found consistent results between a direct measurement of the 
spatial correlation function and a deprojection of the angular correlation function when tested on the same sample of X-ray selected AGN. 
For this reason, we will investigate use of the angular correlation function in this work. 




%%%%%%%%%%%%%%%%%%%%%%%%%%%%%%%%%%%%
% DATA 
%%%%%%%%%%%%%%%%%%%%%%%%%%%%%%%%%%%%%
\section{DATA} 
   Our data consist of 22,000 AGN selected from the SWIFT AGN \& Cluster Survey (SACS) \citep{dai_2015}. This medium-depth Survey
    compiled X-ray sources detected in 739 SWIFT GRB fields. AGN candidates were selected as non-extended X-ray sources in the SACS catalog and matched
    to sources in the Mid IR \textit{WISE} survey. The  total sky coverage of the SWIFT GRB fields ($~125 \ deg^2$) and medium depth of the survey provides a bridge between previous studies of AGN clustering 
    in deep pencil-beam surveys and shallower wide-field surveys. Of the total AGN sample, our dataset contain 17,000 AGN detected in the soft-band (0.2 - 2 keV) and 
    10,000 detected in the in hard band (2 - 8 keV). The combination of broad X-ray spectral coverage, available MIR colors, large sample size, and survey depth provides us 
    a unique opportunity to investigate the clustering of both obscured and unobscured AGN.
    
%%%%%%%%%%%%%%%%%%%%%%%%%%%%%%%%%
% METHODOLOGY 
%%%%%%%%%%%%%%%%%%%%%%%%%%%%

\section{METHODOLOGY}
A quantity of interest in studying the large-scale structure of AGN is the number overdensity parameter, defined as
\begin{align}
    \label{eq:delta}
    \delta = \frac{n-\bar n}{\bar n}
\end{align}
where $\bar n $ is the average number of AGN found in the sky. It has long been known that AGN do not trace the total
distribution of matter in the universe \citep{bardeen}, so  it is appropriate to quantify this with a bias parameter $b$:
\begin{align}
    \label{eq:bias}
    b = \frac{\delta_{AGN}}{\delta_{DM}}
\end{align}
Where $\delta_{DM}$ is the overdensity of dark matter halos. In this work, we will use the angular correlation function to infer 
the real-space clustering properties of our AGN sample. Similar to the definition of the spatial correlation function, the two-point angular correlation function measures the excess 
 probability that two sources within infinitesimal solid angles $d\omega_1$ and $d\omega_2$ are located at a given angular separation from one another. 
 At small scales, the two-point angular correlation function has been shown to follow a simple power-law behavior:
 \begin{align}
    \label{p_law}
    W(\theta) = \left(\frac{\theta}{\theta_0} \right)^{1-\gamma}
 \end{align}
 Where $\theta_0$ is the characteristic angular clustering scale $\gamma$ is the power-law index.  
 There are multiple methods that can be used to estimate the angular correlation function at various angular scales.
In this paper we use the Hamilton Estimator from \citet{hamilton}:

\begin{align}
    \label{eq:hamilton_corr}
    W(\theta) = N \frac{DD \times RR}{DR^2}-1
\end{align}
Where $DD$, $RR$, and $DR$ are the number of data-data, random-random, and data-random pairs between a real an simulated data set
in a given bin of angular separation, respectively. The uncertainty for this estimator is given by

\begin{align}
    \label{eq:hamilton_sig}
    \sigma_W = \frac{1+W}{\sqrt{DD}}
\end{align}

Our process for measuring the correlation function is done in two parts: First, we
iterate through various exposure maps and use the flux limit of each pixel to generate random
sources for that field (see below). Next, we tally the  $DD$,$RR$, and $DR$ pair counts within each field. So long as there are more than a user-specified
number of sources in the field (for this analysis, we assume $N_{cut} =2$), the correlation function is calculated 
for that field according to Eq.\ref{eq:hamilton_corr} for a user-specified binning scheme. In this work, 
we take logarithmically spaced bins to a maximum of 1200 arcseconds (1" = $\frac{1}{3600}^\circ$). Pair counts at a separation smaller than 18" would 
not be resolvable due to the SWIFT XRT's point-spread function (PSF) size of 17.35", so we do not consider angular scales smaller than this in
our analysis. Finally, we clean out bins with no data and evaluate
the weighted average correlation function in each bin according to 

\begin{align}
    \label{eq:weight_avg}
    \bar W = \frac{\sum_{i=1}^n\left(\frac{W_i}{\sigma^2_i}\right)}{\sum_{i=1}^{n}\frac{1}{\sigma_i^2}}
\end{align}

The standard error of each weighted mean is given by 

\begin{align}
    \label{eq:weight_err}
     \sigma_{\bar W_i} = \sqrt{\frac{1}{\sum_{i=1}^{n}\sigma_i^{-2}}}
\end{align}




\subsection{RANDOM CATALOG GENERATION}
To generate the random catalogs for this analysis, we follow a modified version
 of the ``sensitivity map'' procedure outlined in \cite{koutoulidis_2}.
We begin by reading in the SWIFT XRT exposure maps for each SACS GRB field. The exposure time of a given pixel in the map can be directly related to a flux limit associated
with that exposure time and therefore a sensitivity. Once we determine the flux limit of every pixel in the exposure map, we use the
differential number counts statistic to infer the expected number of sources per pixel. The differential number counts
are given by 
\begin{align}
    \label{eq:dnds}
    \frac{dN}{dS} = \bigg\{^{K(S/S_{ref})^{-a} \ \  \ \ \ (S \leq f_b)}_{K(f_b/S_{ref})^{b-a} \ \ \ \ (S > f_b)}
\end{align}
where $f_b$ is flux associated with the power law break, $S_{ref}$ is a reference flux, 
and $K$, $b$, and $a$ are fit parameters associated with the normalization and power law indices, respectively.
We adopt the values  of the power law fit from \cite{dai_2015} for this analysis. To find the cumulative number counts $(N>S)$, we 
integrate Eq. \ref{eq:dnds} from the flux limit of the pixel to infinity. This can be analytically expressed as

\begin{figure}[!ht]
    \centering
    \includegraphics[scale = 0.8]{figs/ns.png}
    \label{fig:ns}
    \caption{Cumulative source count vs. Flux Limit for the simulated AGN (red dashed line) 
    from integrating Eq. \ref{eq:dnds}. This plot shows the number of AGN (per square degree) that should
    be detectable for a given flux limit. Given that the the SWIFT XRT has a pixel scale of 2.36" per pixel, we can 
    use this graph to estimate the number of AGN that should appear in a random image.}
\end{figure}

\begin{align}
    \label{eq:nss}
    N( \ > \ S ) = \bigg\{^{-K\left(\frac{1}{-a+1}\right)\left(\frac{1}{S_{ref}}\right)^{-a}f_b^{-a+1}s^{-a+1} \ \ \ \ \ \ \ \ \ \ \ \ \ (S \leq f_b)}_
    {-K \left(\frac{f_b}{S_{ref}}\right)^{b-a}\left(\frac{1}{-b+1}\right)\left(\frac{1}{S_{ref}}\right)^{-b}f_b^{-b+1}s^{-b+1} \ \ (S > f_b)}
\end{align}

The relative pixel values of each exposure map are used to make a weight-map that describes the 
likelyhood of a given pixel in a synthetic image having a source. To place the sources inferred by the integrated differential source count,
we draw $N$ values from a uniform distribution between zero and the total weight of the 
weight-map. We then iterate through each pixel of the exposure map and compare the value of the draw to that pixel's
associated boundaries in the weight-map. If the draw falls within the weight-map boundaries for that pixel,
we update the value of the associated pixel to be non-zero in an associated random image. Finally, we record the X and Y positions of all 
non-zero pixels in the random image and move on to the next field.


As a test of the accuracy of our integration, we plot the number of sources against the mean exposure map value
(a proxy for exposure time) for both the real and simulated sources. As evident in Fig \ref{fig:exp_counts}, the number of random sources scales
comparably with the number of real sources. Thus, our technique for estimating the number of sources in each field by integrating
Eq. \ref{eq:dnds} is reliable.

\begin{figure}[!ht]
    \centering
    \includegraphics[scale = 0.8]{figs/exp_counts.png}
    \label{fig:exp_counts}
    \caption{Number of sources vs Mean Exposure Map Value for all 739 fields. The number of simulated sources
    evolves similarly to the number of real sources with increasing exposure time, suggesting our analytic integration of the power-law
    differential source count is accurate. Note that the scatter of the data is larger than the simulation. }
\end{figure}


\begin{figure}[!ht]
    \centering
    
    \plottwo{figs/exp_map.png}{figs/rand_img.png}
    \label{fig:exp_map}
    \caption{A sample exposure map from one of the 739 SACS grb fields (left) and the associated random image (right). 
    The number of random sources is 100 times the number of real sources observed in this particular field. Each source
    is color-coded by a flux drawn from the observed log N - log S distribution. }
\end{figure}

\section{TESTING THE 2PACF ALGORITHM}
In order to make sure the random catalogs generated are accurate and we are calculating the correlation function correctly,
we first test the correlation function algorithm on two separately generated random catalogs, i.e., one of the two random catalogs 
is arbitrarily labeled as ``data" and compared against the other. 
The expected result for this procedure is  $W(\theta) \approx 0$ for all angular scales.  As shown in Fig \ref{fig:corr_rand}, our algorithm 
produces a result consistent with this  for most angular scales, suggesting that any signal that
arises in a calculation with the real dataset will be physical.
\begin{figure}[!ht]
    \centering
    \includegraphics[scale = 0.8]{figs/corr_rand.png}
    \label{fig:corr_rand}
    \caption{Two-Point Angular Correlation Function evaluated between
    two random catalogs. The result is mostly consistent with no correlation except 
    on very small angular scales.}
\end{figure}


\section{RESULTS: 2PACF ON SACS DATASET}
\subsection{Complete Sample}
After determining that our algorithm was preforming as expected, we repeat the correlation analysis above 
using the SACS AGN catalog as our dataset. Each field contains an average of 30 AGN, though we disregard fields 
that contain less than two real or rand points. After calculating the weighted average correlation function,
we use the software package \textit{Sherpa} \citep{sherpa} to fit a powerlaw to the measurement. We find a slope of 
$\gamma = 2.1 \pm 0.52$ and a clustering amplitude of $\theta_0 = 8.24 \pm 2.21$ for the complete dataset. The clustering
amplitude is consistent with the findings of  \citet{elyiv} while our slope more closely matches that of \citet{ebrero}, whose softband analysis
featured a slightly larger AGN catalog than our own. 

\begin{figure}[!h]
    \centering
    \includegraphics[scale = 1]{figs/corr_all.png}
    \label{fig:corr_all}
    \caption{Two-Point Angular Correlation Functions for the the complete SACS AGN sample. Though there are many
    neagtive values for small angular scales, we find a powerlaw behavior consistent with 
    previous results in the literature }
\end{figure}

\subsection{Obscured vs Unobscured Subsample}
The canonical picture of AGN unification attributes the differences in observed AGN spectral type (Type I or Type II)
to the viewing angle of the observer \citep{antonucci:1985}. If this model is correct, then both Type I and Type II AGN should 
have similar host halo mass distributions and therefore cluster similarly on the sky \citep{mitra:2018}.To test this, we divide our 
samples of SACS AGN into ``MIR red'' and ``MIR blue'' using the scheme from \cite{dai_2015}
We label AGN with $W1 - W2 \leq 0.35$ mag MIR blue and those with color $W1 - W2 > 0.35$  as MIR red. Using this criteria, the fraction of obscured 
and unobscured AGN are $f_{blue} = 0.53 , f_{red} = 0.47 $, respectively.

Our procedure for investigating the color dependence of the angular correlation function is as follows:
We construct a new random catalog following the original procedure, but this time generate two random images per field. The number of random sources in each image
is the expected result from the procedure in section 4 multiplied by $f_{blue}$ and $f_{red}$ for the blue and red AGN, respectively. Next we repeat the correlation analysis
described in Section 3 for the two subsamples separately. For the blue subsample, we find a powerlaw slope of $\gamma_b = 1.95 \pm 0.21$ and a clustering amplitude of $\theta_{0,b} = 9.00 \pm 11.25 $. For the 
red AGN subsample, we measure $\gamma_r = 1.54 \pm 0.32 $ and $\theta_{0,r} = 0.82 \pm 2.94$ 

\begin{figure}[!h]
    \centering
    \includegraphics[scale = 0.75]{figs/colors_hist.png}
    \label{fig:hist_color}
    \caption{Histogram showing the ratio of $N_{rand}$ to $N_{data}$ for all fields with $N_{data},N_{rand}> 2$. To reduce shot noise, we multiply
    the expected number of random sources by a factor  two. The means of both distributions are similar but differ slightly due to the different fractions
    of MIR red and MIR blue AGN in our dataset.  }
\end{figure}


\begin{figure}[!h]
    \centering
    \includegraphics[scale = 0.9]{figs/corrfunc_color.png}
    \label{fig:corr_color}
    \caption{Two-Point Angular Correlation Functions for the MIR Red and MIR Blue AGN subsamples. The clustering amplitude and slope are
    slightly larger for the MIR Blue AGN in our sample. }
\end{figure}

\begin{deluxetable}{cccc}

    %% Keep a portrait orientation
    
    %% Over-ride the default font size
    %% Use Default (12pt)
    
    %% Use \tablewidth{?pt} to over-ride the default table width.
    %% If you are unhappy with the default look at the end of the
    %% *.log file to see what the default was set at before adjusting
    %% this value.
    
    %% This is the title of the table.
    \tablecaption{Best Fit Values}
    
    %% This command over-rides LaTeX's natural table count
    %% and replaces it with this number.  LaTeX will increment 
    %% all other tables after this table based on this number
    \tablenum{1}
    
    %% The \tablehead gives provides the column headers.  It
    %% is currently set up so that the column labels are on the
    %% top line and the units surrounded by ()s are in the 
    %% bottom line.  You may add more header information by writing
    %% another line between these lines. For each column that requries
    %% extra information be sure to include a \colhead{text} command
    %% and remember to end any extra lines with \\ and include the 
    %% correct number of &s.
    \tablehead{\colhead{Sample} & \colhead{$\theta_0 \ (")$} & \colhead{$\gamma$} & \colhead{$\chi^2_\nu$} } 
    
    %% All data must appear between the \startdata and \enddata commands
    \startdata
    SACS All & 8.14$\pm$3.11 & 2.07$\pm$0.087 & 0.75 \\
    MIR Red & 32.02$\pm$11.26 &  2.16$\pm$0.058 & 1.08 \\
    MIR Blue & 28.65$\pm$8.35  & 2.11$\pm$0.049 & 1.57 \\
    \enddata
    
    %% Include any \tablenotetext{key}{text}, \tablerefs{ref list},
    %% or \tablecomments{text} between the \enddata and 
    %% \end{deluxetable} commands
    
    
    %% No \tablerefs indicated
    
    \end{deluxetable}

    \section{DEPROJECTING THE ANGULAR CORRELATION FUNCTION}
    Our observed clustering signal is a projection of the true three dimensional clustering of the AGN population.
    As discussed above, we can infer the true clustering properties by deprojecting our angular measurement. Our primary tool for 
    investigating the real space properties of AGN from the 
    angular correlation function is Limber's Inversion Equation \citep{limber}. 
    From \citet{koutoulidis}, this integral equation can be expressed in terms of 
    observable properties as :
    
    \begin{align}
        \theta_0^{\gamma-1} = H_\gamma x_0^\gamma \int_0^{\infty}\left(\frac{1}{N} \frac{dN}{dz} \right)^2
        \frac{d_A(z)^{1-\gamma}}{cd\tau(z)/dz} \tilde b^{2} (z) \tilde D^{2+n}(z) dz
    \end{align}
    
    where $H_\gamma$ is related to the gamma function and the powerlaw index of the measured angular correlation function by
    
    \begin{align}
        H_\gamma = \frac{\Gamma\left( \frac{1}{2} \right)\Gamma \left( \frac{\gamma - 1}{2} \right)}{\Gamma \left(\frac{\gamma}{2}\right)}
    \end{align} 
    The remaining terms are various cosmological parameters related to the angular diameter distance ($d_A$), growing mode of linear perturbations ($\tilde D$) and the lookback time ($ d\tau / dz$).
    Of interest to this analysis is the normalized bias parameter $\tilde b$, defined in \cite{koutoulidis} as the bias parameter $b$ normalized
    to its value at $z = 0$:
    
    \begin{align}
        \tilde b(z) = \frac{b(z)}{b(0) }
    \end{align}
    
    Where the AGN bias from Eq. \ref{eq:bias} can be related to underlying cosmological parameters according to 
    \begin{align}
        b(z) = 1 + \frac{b_0 - 1}{D(z)} + C_2 \frac{J(z)}{D(z)}
    \end{align}
    
    The $b_0$ and $C_2$ terms are ultimately related to the host dark matter halo mass by
    
    \begin{align}
        b_0(M_h) = 0.857\left[1+ \left(\frac{\Omega_{m,0}}{0.27} \frac{M_h}{10^{14}h^{-1}M_{\odot}}\right)^{0.55}\right]
    \end{align}
    
    \begin{align}
        C_2(M_h) = 1.105\left[1+ \left(\frac{\Omega_{m,0}}{0.27} \frac{M_h}{10^{14}h^{-1}M_{\odot}}\right)^{0.255}\right]
    \end{align}
    where $\Omega_{m,0}$ is the fractional matter density in the local universe and $h$ is the ``ignorance parameter" \cite{hogg} related to the Hubble Constant as 
    
    \begin{align}
        H_0 = 100h \ \text{km} \ \text{s}^{-1} \  \text{Mpc}^{-1}
    \end{align}
    In the future it may be worth preforming our own fits for the characteristic host halo mass, but for now we'll adopt the value assumed by \citet{koutoulidis} ($M_h = 1.3 \times 10^{13} M_{\odot})$
    To deproject our measurement, we need to know the average number of AGN contained within the average solid angle subtended 
    by our fields at each infinitesimal redshift slice:
    
    \begin{align}
        \label{eq:dndz}
        \frac{dN}{dz} = \Omega_s d_A(z)^2 (1+z)^2 \phi(z) \left(\frac{c}{H_0}\right)\frac{1}{E(z)}
    \end{align}
    Where $\phi(z)$ describes the probability that a source at a particular redshift will be observed. This quantity depends 
    on the redshift distribution of the AGN in our sample. Since we don't have redshift information we must assume an X-ray luminosity function 
     $\Phi(L,z)$ to infer $\phi(z)$ according to
     \begin{align}
         \label{eq:phi}
         \phi(z) = \int_{L_{min(z)}}^{\infty} \Phi(L,z) dL 
     \end{align}
    
     where $L_{min(z)}$ is the minimum luminosity we can observe at a particular redshift. We can determine the minimum detectable luminosity at each redshift slice
    by preforming a $k$-correction to account for the expansion of the universe on the standard flux-luminosity-distance relationship:

    \begin{align}
        \label{eq:flux_lim}
        \bar f_{\nu,min} = \frac{L_{\nu,min}(1+z)^{1-\alpha}}{4\pi D_L^2}
    \end{align}
    Where $ \bar f_{\nu,min}$ is the mean flux limit of our survey and $D_L$ is the luminosity distance to 
    a source at a redshift $z$.
     In the future we'll adopt more modern values for the AGN luminosity function, but for now we'll assume the same Luminosity and Density Evolution  (LADE) luminosity function used by \citet{koutoulidis} \citep{aird:2010}. 
    A sample of the luminosity function (evaluated at a random redshift of z = 0.1) is shown below in Fig. 
    
    \begin{figure}[!ht]
        \centering
        \includegraphics[scale = 1.4]{figs/lum_func.png}
        \label{fig:lum_func}
        \caption{The LADE Luminosity function (with parameters from \citet{aird:2010}) used for deprojecting the angular correlation function in our analysis, shown here evaluated at a redshift of z = 0.1
        This function can be integrated to infer the number of observable AGN density (per $\text{cm}^{-3}$) for a given minimum observable luminosity.}
    \end{figure}
    
    \begin{figure}[!ht]
        \centering
        \includegraphics[scale = 1.2]{figs/z_dist.png}
        \label{fig:z_dist}
        \caption{The AGN redshift selection function of our sample  inferred by integrating the above luminosity function over all redshifts. 
        The shape of our selection function matches that of \citet{ebrero}. }
    \end{figure}
    
    For our slope of $\gamma = 2.07$ and we find a spatial clustering length at 
    $z = 0$ of $x_0 = 10.64 h^{-1} \  \text{Mpc}^{-1}$. We can find the clustering length at a particular redshift 
     (in comoving coordinates) according to 
    
    
    \begin{align}
        \label{eq:clustering_length}
        x_0(z) = x_0(0) \tilde D^{2+n/\gamma}(z) \tilde b^{2/\gamma}
    \end{align}
    Normalizing our clustering length the mean of our redshift distribution in Fig. (\ref{fig:z_dist}), $(z \approx 0.75)$, we find a comoving clustering length 
    of $x_0 = 15.39 h^{-1} \text{Mpc}^{-1}$.


\section{DISCUSSION}
For the complete dataset, our measurement of the correlation function is in strong agreement
with other soft-band X-Ray surveys in the literature (see Fig. (\ref{fig:comp})). The calculated angular clustering amplitude 
sits between the measurement of \cite{ebrero}, whose analysis featured a similarly sized dataset, and \cite{elyiv}. The 
measured slope of our dataset agrees strongly with that of the slope found by \cite{ebrero} in the softband. When splitting sample but MIR color, we find similar values for the slope 
of the angular correlation function for both obscured ang unobscured AGN. The best fit clustering amplitude is notably larger for the two subsamples compared to the complete sample 
(but more in line with the larger amplitudes found by \citet{ebrero} in the soft band). The slopes and clustering amplitudes of the two subsamples are in close agreement at most angular scales.
consistent with the findings of the hardness-ratio analysis of  \cite{ebrero} and unification models of AGN.
Thus a separate metric for splitting the complete sample into obscured and unobscured AGN found no strong difference
between the two subsamples' clustering properties.

\begin{figure}[!ht]
    \centering
    \includegraphics[scale = 0.8]{figs/results_compar.png}
    \label{fig:comp}
    \caption{A summary of our contribution to the literature. \textit{Top}: Measurements for $\gamma$ and $\theta_0$
    from other soft-band X-ray survey measurements. The canonical galaxy value of  $\gamma = 1.8$ is plotted as the dashed black line,
    indicating that X-ray selected AGN are not well characterized by this slope. \textit{Bottom} The above results when fit to a value of 
    $\gamma = 1.8$. From Limber's Inversion, real-space and angular correlation functions are related by $w(\theta) \propto \xi^{\gamma - 1}$. 
    Reporting the powerlaw fit of the angular correlation function with fixed slope of $\gamma - 1 = 0.8$ is common 
    in the literature. 
    }
\end{figure}

The canonical picture of AGN unification discussed above is also known as ``unification by orientation.''  Although this has been thoroughly investigated
in the literature, there are are some open issues with this unification scheme that have yet to be explained (see \citet{netzer:2015} for a comprehensive review 
of AGN unification schemes). An alternative unification scheme discussed in the literature is ``unification by evolution,'' where different AGN 
spectral types are really the same phenomena observed at different epochs in their lifetimes, regardless of viewing angle. For this unification scheme, one would 
expect the distribution of different AGN types to evolve with redshift due to the increasing look-back time with redshift. 
Since our measurement of the correlation function relies only
on the angular positions of obscured and unobscured AGN and not on their redshift, we would be unable to identify differences 
in the clustering of these two populations for this unification scheme. 


Direct measurements of the clustering of optically selected AGN have found a physical clustering lengths on the order of $r_0 \approx 5.4 - 8.6 h^{-1} \text{Mpc}^{-1}$ \citep{akylas:2000,croom:2002,grazian:2004}. 
Similar surveys of X-ray selected AGN have inferred a slightly higher spatial clustering length in the range of $r_0 \approx 7.5 - 9.8 h^{-1} \text{Mpc}^{-1}$ \citep{basilakos:2008,miyaji}. To compare our result to 
those reported in the literature, we'll express the clustering length in proper coordinates according to 
\begin{align}
    \label{eq:prop_coord}
    r_0 = \left(\frac{x_0}{1+z}\right)
\end{align}
Our estimate of the comoving clustering length at $z = 0.75$ translates to a physical distance of $r_0 = 8.79 h^{-1} \text{Mpc}$, in excellent agreement with the above measurements 
as well as the inferred spatial clustering length of \citet{ebrero}.
\section{CONCLUSION}

In this work we measured the two-point angular correlation function
for 22,000 X-ray selected AGN. Our dataset provides a bridge 
between previous high depth narrow surveys and
shallow wide-field soft-band x-ray surveys. We find an angular clustering 
amplitude and slope consistent with previous results for the soft-band in the literature, 
suggesting the clustering of AGN behaves as expected for the AGN population not sampled by previous surveys.
We find no statistically significant difference in the clustering amplitude or 
slope for the obscured or unobscured subsamples, consistent with unification models of AGN.

As a final demonstration of the utility of the angular correlation function, we deproject our measurement to find a clustering length consistent with the literature.
Additional analysis prior to publication may include a comparison of the clustering 
for different flux-limited subsamples as well as a separate clustering analysis for available SACS hard band (2-8 kev) data. As a followup to 
\citet{migkas}, who found an anisotropy in the galaxy cluster $L_x - T$ relation, it may also be worth investigating the directional dependence of 
the clustering properties of our sample as a long term goal.



%%%%%%%%%%%%%%%%%%%%%%%%%%%%%%%%%%%%%%%%%%%%%%%
% Bibliography 
%%%%%%%%%%%%%%%%%%%%%%%%%%%%%%%%%%%%%%%%%%%%%%%

\bibliography{refs}




\end{document}


